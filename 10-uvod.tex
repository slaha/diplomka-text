\documentclass[dp.tex]{subfiles}


\begin{document}
\chapter*{Úvod}
\addcontentsline{toc}{chapter}{Úvod} 
\label{chap:uvod} 
%Lingvistika je věda, která se, jak její český název \textit{jazykověda} napovídá, zabývá studiem a výzkumem jazyka. Vzhledem k množství existujících jazyků a ke komplexnosti jazyka je lingvistika dále dělena na velké množství dílčích oborů dle toho, který z jazyků nebo kterou část jazyka zkoumá.

%Jako \textit{matematická lingvistika} je označována pomezní disciplína spojující lingvistiku, matematiku a informatiku. Matematická lingvistika se začala formovat v padesátých letech dvacátého století. Za její počátek bývá uváděn VIII. mezinárodní lingvistický kongres konaný roku 1957 v Oslu. Nicméně k prvním spojením jazykovědy s matematikou docházelo již dříve, a to zejména v rámci metod kvantitativních soustředěných na pojem \textit{frekvence}. 

%S rozvojem výpočetní techniky ve druhé polovině dvacátého století docházelo ke stále většímu využití počítačů ve všech oborech lidské činnosti. Ani matematická lingvistika nebyla výjimkou. \textit{Počítačová lingvistika} (též \textit{strojová}, anglicky \textit{computational linguistics}) je jednou z disciplín matematické lingvistiky. Počítačová lingvistika využívá poznatků kvantitativní i algebraické lingvistiky, k jejichž aplikaci je využíváno počítačů. \textit{Zdroj: \cite{Cerny1996}}.

%Charakteristiky použité k určování autorství v rámci bakalářské práce nebudou v této práci popisovány. Jejich podrobný popis je obsažen v původní bakalářské práci, případně ve zmiňovaném článku. V diplomové práci budou čtenáři podrobně seznámeni s denotační analýzou. Popsán bude také vytvořený počítačový program umožňující provádění analýz textu, a to jak z technického, tak uživatelského pohledu. Poslední kapitola je pak zaměřena na samotnou analýzu a porovnání osmnácti českých překladů Poeova Havrana.

Nový úvod: popsat obsah této DP.
\end{document}
