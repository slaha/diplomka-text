\documentclass[dp.tex]{subfiles}
\begin{document}
\chapter{Denotační analýza}

Denotační analýza patří mezi analýzy referenčního typu, neboť jejím smyslem je rozložení textu na jednotky (slova, fráze, \ldots) a rozdělení těchto jednotek do skupin zvaných \textbf{hřeby}. Každý hřeb spolu sdružuje jednotky, které mezi sebou z hlediska významu souvisí. Hřeb tedy sdružuje všechny jednotky vztahující se k jednomu a tomu samému subjektu. Například při analýze jakéhokoliv překladu básně Havran bude při denotační analýze nalezen hřeb, který bude sdružovat jednotky související s havranem.

\section{Dělení textu do hřebů}

Při denotační analýze nejprve text rozdělíme do hřebů. Na základě zvolených kritérií je třeba slova zařadit do příslušných hřebů. Tato kritéria nejsou doposud pevně určena. Vytvoření kritérií tak, aby pokryla všechny případy, které mohou nastat, je velmi komplexní úkol. Je však vhodné vytvořit si vlastní soubor těchto kritérií tak, aby pokrýval případy, které se vyskytují ve zkoumaných textech. Tím bude zachována stejná metodika pro všechny texty a následné porovnávání výsledků tak nebude zkresleno rozdílnými pravidly, jež byly při rozdělování použity.

Při tvorbě kritérií je možné inspirovat se pravidly, které uvádí Altman v \cite{Wimmer2003} na straně 298:
\begin{quote}
\begin{itemize}
\item Podstatná jména:
\begin{enumerate}
\item deklinace nemění  \glslink{denotace}{denotaci}, proto je ignorována,
\item zdrobněliny se přiřazují k původnímu \glslink{substantivum}{substantivu}.
\end{enumerate}
\item Slovesa:
\begin{enumerate}
\item osobní koncovky, jednotné a množné číslo jsou relevantní, odkazují-li na entitu v textu. Sloveso v množném čísle odkazuje na všechny hřeby, jichž se sloveso týká.
\item Čas je irelevantní: \uv{pravím}/\uv{pravil jsem}/\uv{budu pravit} vytváří hřeby \uv{pravit} a \uv{já}. Je však možné vytvářet i hřeby pro různé časy.
\item Sloveso a zvratná forma jsou identické: \uv{ozvalo}/\uv{ozvalo se}.
\item Slovesný \glslink{slovesnyvid}{vid} je irelevantní.
\item Nedokonavé a dokonavé sloveso s předponou je možno rozlišovat nebo nerozlišovat v závislosti na významu.
\item Analytické formy tvoří jeden celek, i když jsou od sebe oddělené.
\item Pozitivní a negativní tvar slovesa jsou identické.
\end{enumerate}
\item Přídavná jména:
\begin{enumerate}
\item rod a pád jsou ignorovány.
\end{enumerate}
\item Zájmena:
\begin{enumerate}
\item rod a pád jsou ignorovány.
\end{enumerate}
\item Všeobecná pravidla:
\begin{enumerate}
\item V případě nejistoty platí, že je lepší vytvořit méně hřebů než více.
\item Je možné (uznáme-li to za vhodné) sloučit do jednoho hřebu různé slovní druhy (např. \uv{klepot} a \uv{klepá}).
\end{enumerate}
\item Některé fráze mohou tvořit jeden element (např.\uv{nikdy víc}), jiné zase mohou odkazovat na jiný nesložený element (např. \uv{černý pták} znamená vlastně \uv{havran}).
\end{itemize}
\end{quote}

V případě ručního dělení textu do hřebů je vhodné připravit text tak, že každému denotačnímu elementu je přiděleno pořadové číslo. Obsahuje-li slovo více denotačních elementů, jsou jejich pořadová čísla oddělena svislou čarou \uv{|}. Tato úprava je znázorněna na první sloce překladu Havrana od Ivana Slavíka.

\begin{samepage}
\begin{Verbatim}[commandchars=+\[\]]
 1|2   3   4    5       6   7|8   9|10   11   12
Bylo pusto o půlnoci, když jsem hloubal bez pomoci,

 13|14   15    16   17   18     19     20
luštil marně živou mocí staré svazky kajícné,

 21|22   23|24  25|26   27   28  29 30  30     31
nedřímal jsem, nebděl  ani, když tu ťuk ťuk znenadání

 32|33       34         35     36  37  38
ozvalo se zaklepání, zaklepání a  víc  ne.

 39  40  41|42 43 44|45  46     47       48
„Kdo mě shání, co chce  toto zaklepání kajícné?

49|50  51  52  53  54  55 56
Je     to host a  nic víc ne.“
\end{Verbatim}
\end{samepage}

Pokud by analyzovaný text obsahoval denotační element složený z více slov (například výše vzpomínané \uv{nikdy víc}), bylo by tomuto denotačnímu elementu v upraveném textu přiřazeno pouze jedno pořadové číslo a celý denotační element by byl podtrhnut:
\begin{Verbatim}[commandchars=+\[\]]
50|51   52        53
Praví havran +underline[+uv[Nikdy víc.]]
\end{Verbatim}

V první sloce se nachází 56 denotačních elementů, které je třeba rozdělit do jednotlivých hřebů. Při ručním dělení je vhodné psát jednotlivé hřeby pod sebe v pořadí, v jakém se vyskytly v textu včetně jejich pořadového čísla. Nutno podotknout, že dělení textu do hřebů je časově náročná činnost, jejíž složitost narůstá s délkou analyzovaného textu.

Software, který vznikl v rámci této diplomové práce, obsahuje snadno ovladatelné uživatelské rozhraní, které dělení textu do hřebů co možná nejvíce usnadňuje.

\subsection{Další typy hřebů}
Hřeby, které vznikly analýzou textu, je možné transformovat na
\begin{enumerate}
\item \textbf{Množinový hřeb}\\
Množinový hřeb je takový hřeb, v němž se uvedou pouze unikátní denotační elementy. Vyskytují-li se v hřebu denotační elementy související s deklinací, jsou seskupeny do kategorie např. \uv{1. osoba singuláru}, \uv{2. osoba singuláru} atd.
\item \textbf{Poziční hřeb}\\
V pozičním hřebu jsou denotační elementy znázorněny pouze svým pořadovým číslem.
\end{enumerate}

\section{Charakteristiky}

Texty zachycují určitou skutečnost, kterou čtenář při jejich čtení \uv{rekonstruuje}. Skutečnost je v textu zachycena formou pojmů. Aby čtenář text pochopil tak, jak autor zamýšlel, je třeba, aby byl v textu zachycen vztah mezi různými pojmy.

Denotační analýza zkoumá text právě z tohoto hlediska -- jak spolu jednotlivé v textu se vyskytující elementy souvisí. Na základě denotační analýzy lze odvodit následující charakteristiky textu.

\subsection{Jádro textu}
Jádro textu obsahuje ty hřeby, v nichž se vyskytují nejméně dva odlišné kmeny\footnote{Kmenem se v lingvistice označuje ta část slova, která se v různých tvarech téhož slova (u flektivních slovních druhů) nemění. Mnohdy bývá totožná s kořenem (morfémem nesoucím základní významovou (lexikální) část slova. \url{http://cs.wikipedia.org/wiki/Kmen_(mluvnice)} }.

\subsection{Velikost hřebu}
Velikost (též rozsah) hřebu je definována jako počet elementů v daném hřebu. Velikost hřebu %TODO sem dát hřeb
HŘEB je $X$.

\subsection{Kardinální číslo jádra}
Kardinální číslo jádra se značí $|\kern 2pt.\kern 2pt|$. Je dáno součtem velikostí hřebů, které jsou součástí jádra textu.

\subsection{\uv{Lokálnost} jádrových hřebů}
\uv{Lokálnost} jádrových hřebů (značena $T$) zobrazuje dominantnost určitých hřebů v textu.  Jedná se o poměr velikosti hřebu ke kardiálnímu číslo jádra:
\begin{equation}
T_\text{(hřeb)}=\frac{\text{velikost hřebu}}{|\kern 5pt.\kern 5pt|}.
\end{equation}
Viz \cite[str. 302]{Wimmer2003}. 

\subsection{Deterministické jádro}
%TODO sem dát hřeb "zaklepání"
Deterministické jádro souvisí s problematikou  \glslink{tema_rema}{témy a rémy}. Do deterministického jádra patří všechny hřeby, které na sebe odkazují. Například vidíme, že se ve hřebu \uv{zaklepání} nachází slovo \uv{(ozva)lo}. V deterministickém jádru proto bude obsažen jak hřeb \uv{zaklepání}, tak hřeb \uv{ozvat}.


\subsection{Kompaktnost textu}
Kompaktnost textu určuje, jak moc je text zaměřen na jeden (hlavní) motiv. To přímo úměrně souvisí s počtem hřebů, které se v textu vyskytují. Kompaktnost textu $K$ je číslo ležící v intervalu $\langle0, 1\rangle$. Čím je vyšší, tím je text kompaktnější. Naopak čím nižší je hodnota $K$, tím více je text rozvětven. Vzorec pro výpočet hodnoty $K$ je v \cite[str. 303]{Wimmer2003} definován jako
\begin{equation}
K=\frac{1-\frac{N}{L}}{1-\frac{1}{L}},
\end{equation}
kde 
\begin{align*}
      N & \text{ je počet v textu nalezených hřebů,}\\
      L & \text{ je počet denotačních pozic v textu.}
\end{align*}     

\subsection{Centralizovanost textu}
Centralizovanost textu ($R$) vyjadřuje sílu, jakou se denotace koncentruje na centrální hřeby. Je závislá na počtu výskytů jednotlivých hřebů. Spočítá se (viz \cite[str. 303]{Wimmer2003}) jako
\begin{equation}
R=\frac{1}{n^2}\sum{ \abs{H_i}^2 \times f_i},
\end{equation}
kde 
\begin{align*}
      \abs{H_i} & \text{ je velikost hřebu,}\\
      f_i & \text{ je počet hřebů s velikostí \abs{H_i},}\\
      n&=\abs{H_i}\times f_i.
\end{align*}     

Pomocí MacIntoshova indexu je možné spočítat relativní centralizovanost textu. Vzorec výpočtu MacIntoshova indexu je definován v \cite[str. 304]{Wimmer2003} jako
\begin{equation}
R_{rel}=\frac{1-\sqrt{R}}{1-\frac{1}{\sqrt{n}}}.
\end{equation}

\subsection{Difuze hřebu}
Jedná se o relativní rozdíl mezi prvním a posledním výskytem slova z jednoho hřebu v textu. Z toho vyplývá, že počítat difuze má smysl pouze pro hřeby o velikosti alespoň $2$.

Vztah pro výpočet difuze hřebu ($D_{\text{hřeb}}$) je dána jako:
\begin{equation}
D_\text{hřeb}=\frac{\max{\langle\text{hřeb}\rangle}-\min{\langle\text{hřeb}\rangle}}{\text{velikost hřebu}},
\end{equation}
viz \cite[str. 304]{Wimmer2003}.
\subsection{Rozšířené jádro textu}
Rozšířené jádro je charakteristické pro difuzi textu. Patří do něj všechny hřeby, jejichž difuze je menší nebo rovna nejvyšší difuzi jádrového hřebu, přičemž jsou vynechány všechny předložky, spojky, zájmena a citoslovce.

\subsection{Koincidence}
Koincidence je definována jako výskyt dvou hřebů v jedné vymezené části textu (verš, věta, sloka, \ldots). Koincidence je považována za:
\begin{description}
\item[náhodnou,] pokud pravděpodobnost výskytu daných dvou hřebu je dostatečně vysoka,
\item[signifikantní,] v případě, že pravděpodobnost tohoto výskytu je malá.
\end{description}

Hranice mezi \textit{náhodnou} a \textit{signifikantní} koincidencí je označena $\alpha $. Je-li vypočítaná pravděpodobnost dané koincidence menší nebo rovna $\alpha $, je tato koincidence nazývána \textit{signifikantní koincidence na hladině $\alpha $}. Hodnotu $\alpha $ je třeba volit v závislosti na typech textu a velikosti vymezené části textu, v níž je koincidence zkoumána.

Pro výpočet koincidence je třeba upravit zkoumaný text, a to tak, že se slova v textu nahradí pořadovým číslem hřebu. Pro každou dvojici hřebů $A$ a $B$ se pomocí vzorce
\begin{equation}
P(X \geq x)=\sum_{i=x}^{min{\left(  M, n \right) }} \frac{\binom{M}{i}\binom{N-M}{n-i}}{\binom{N}{n}},
\end{equation}
kde 
\begin{align*}
	N & \text{ je počet vymezených částí textu (počet veršů, vět, slok, \ldots),}\\
	M & \text{ je počet vymezených částí, v nichž se vyskytuje hřeb $A$,}\\
	n & \text{ je počet vymezených částí, v nichž se vyskytuje hřeb $B$,}\\
	x & \text{ je počet vymezených částí, v nichž se vyskytuje jak hřeb  $A$, tak hřeb $B$,}
\end{align*}  

vypočítá kumulativní pravděpodobnost hypergeometrického rozdělení. Koincidence je \textit{signifikantní}, pokud platí $P(X \geq x)\leq \alpha$. Čím je zvolená hladina $\alpha$ menší číslo, tím méně hřebů je vzájemně \textit{signifikantních}.

Z vypočítaných pravděpodobností $P(X \geq x)$ je možné sestrojit $\alpha$-graf. V $\alpha$-grafu jsou hřeby zakresleny jako vrcholy, přičemž sousedními se stanou ty vrcholy (=hřeby), jejichž vzájemná pravděpodobnost $P(X \geq x)$ je menší nebo rovna $\alpha$ (tzn. jsou \textit{signifikantní}).

Z toho vyplývá, že volba velikosti hladiny $\alpha$ má vliv na počet hran a komponent ve výsledném $\alpha$-grafu. 

Charakteristikami, které jsou spojeny s koincidencí, jsou:
\begin{description}
	\item[Index nespojitosti] představuje nejsilnější vztah v textu. Je-li hladina $\alpha$ stanovena na hodnotu nižší, než je \textit{index nespojitosti}, bude $\alpha$-graf tvořen pouze izolovanými vrcholy.
	\item[Index neizolovanosti] je pravděpodobnost zaručující, že žádný z vrcholů v $\alpha$-grafu nebude izolovaný. Pokud je hladina $\alpha$ stanovena nad \textit{index neizolovanosti}, každý vrchol $\alpha$-grafu bude incidentní s alespoň jednou hranou.
	\item[Index dosažitelnosti] je taková hodnota $\alpha$, při které je celý $\alpha$-graf tvořen pouze jednou komponentou. To znamená, že $\alpha$-graf je souvislý a tudíž existuje cesta mezi jeho libovolnými dvěma vrcholy.
\end{description}
\end{document}
