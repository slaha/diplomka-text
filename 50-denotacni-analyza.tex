\documentclass[dp.tex]{subfiles}
\begin{document}
\chapter{Denotační analýza}
\label{chap:denotacni_analyza}

Denotační analýza patří mezi analýzy referenčního typu, neboť jejím smyslem je rozložení textu na jednotky (slova, fráze, \ldots) a rozdělení těchto jednotek do skupin zvaných \textbf{hřeby}. Ty jsou pojmenovány po svém objeviteli L. Hřebíčkovi. Každý hřeb spolu sdružuje jednotky, které mezi sebou z hlediska významu souvisí. Hřeb tedy sdružuje všechny jednotky vztahující se k~jednomu a tomu samému subjektu. Například při analýze jakéhokoliv překladu básně Havran bude při denotační analýze nalezen hřeb, který bude sdružovat jednotky související s havranem.

\section{Dělení textu do hřebů}

Při denotační analýze nejprve text rozdělíme do hřebů. Na základě zvolených kritérií je třeba slova zařadit do příslušných hřebů. Tato kritéria nejsou doposud pevně určena. Vytvoření kritérií tak, aby pokryla všechny případy, které mohou nastat, je velmi komplexní úkol. Je však vhodné vytvořit si vlastní soubor těchto kritérií tak, aby pokrýval případy, které se vyskytují ve zkoumaných textech. Tím bude zachována stejná metodika pro všechny texty a následné porovnávání výsledků tak nebude zkresleno rozdílnými pravidly, jež byly při rozdělování použity.

Při tvorbě kritérií je možné inspirovat se pravidly, které uvádí Altman v \cite{Wimmer2003} na straně 298:
\begin{oldquote}
\begin{itemize}
\item Podstatná jména:
\begin{enumerate}
\item deklinace nemění  \glslink{denotace}{denotaci}, proto je ignorována,
\item zdrobněliny se přiřazují k původnímu \glslink{substantivum}{substantivu}.
\end{enumerate}
\item Slovesa:
\begin{enumerate}
\item osobní koncovky, jednotné a množné číslo jsou relevantní, odkazují-li na entitu v textu. Sloveso v množném čísle odkazuje na všechny hřeby, jichž se sloveso týká.
\item Čas je irelevantní: \uv{pravím}/\uv{pravil jsem}/\uv{budu pravit} vytváří hřeby \uv{pravit} a \uv{já}. Je však možné vytvářet i hřeby pro různé časy.
\item Sloveso a zvratná forma jsou identické: \uv{ozvalo}/\uv{ozvalo se}.
\item Slovesný \glslink{slovesnyvid}{vid} je irelevantní.
\item Nedokonavé a dokonavé sloveso s předponou je možno rozlišovat nebo nerozlišovat v závislosti na významu.
\item Analytické formy tvoří jeden celek, i když jsou od sebe oddělené.
\item Pozitivní a negativní tvary slovesa jsou identické.
\end{enumerate}
\item Přídavná jména:
\begin{enumerate}
\item rod a pád jsou ignorovány.
\end{enumerate}
\item Zájmena:
\begin{enumerate}
\item rod a pád jsou ignorovány.
\end{enumerate}
\item Všeobecná pravidla:
\begin{enumerate}
\item V případě nejistoty platí, že je lepší vytvořit méně hřebů než více.
\item Je možné (uznáme-li to za vhodné) sloučit do jednoho hřebu různé slovní druhy (např. \uv{klepot} a \uv{klepá}).
\end{enumerate}
\item Některé fráze mohou tvořit jeden element (např.\uv{nikdy víc}), jiné zase mohou odkazovat na jiný nesložený element (např. \uv{černý pták} znamená vlastně \uv{havran}).
\end{itemize}
\end{oldquote}

V případě ručního dělení textu do hřebů je vhodné připravit text tak, že každému denotačnímu elementu je přiděleno pořadové číslo. Obsahuje-li slovo více denotačních elementů, jsou jejich pořadová čísla oddělena svislou čarou \uv{|}. Analytické formy sloves (např. \enquote{hloubal jsem}) tvoří pouze jeden denotační element. Víceslovné elementy je pro přehlednost vhodné podtrhnout. Tato úprava je znázorněna na první sloce překladu Havrana od Ivana Slavíka.

\begin{Verbatim}[commandchars=+\[\]]
 1|2   3   4    5       6      7|8       9    10
Bylo pusto o půlnoci, když +underline[jsem hloubal] bez pomoci,

 11|12   13    14   15   16     17     18
luštil marně živou mocí staré svazky kajícné,

    19|20      21|22    23  24   25  26  27   28
+underline[nedřímal jsem], nebděl  ani, když tu ťuk ťuk znenadání

  29|30      31         32     33  34  35
+underline[ozvalo se] zaklepání, zaklepání a  víc  ne.

 36  37  38|39  40 41|42  43     44       45
„Kdo mě  shání, co chce  toto zaklepání kajícné?

46|47  48  49  50  51  52 53
Je     to host a  nic víc ne.“
\end{Verbatim}

V první sloce se nachází 53 denotačních elementů, které je třeba rozdělit do jednotlivých hřebů. Při ručním dělení je vhodné psát jednotlivé hřeby pod sebe v pořadí, v jakém se vyskytly v textu včetně jejich pořadového čísla a informace o tom, k čemu se daný hřeb vztahuje. Slovesa typicky obsahují více denotačních elementů --- jeden pro základní tvar slovesa, ostatní odkazují na entity v textu. Například sloveso \enquote{luštil} obsahuje dva denotační elementy (11 a 12). Element s číslem 11 odkazuje na infinitiv slovesa a jako takový je zařazen do hřebu 10. Koncovka \enquote{l} však odkazuje na vypravěče, a proto je zařazena do hřebu číslo 6. Do závorek $(\quad )$ je v takovém případě vepsána ta část  elementu, která do hřebu nepatří, ale je s ním spojena.

Nutno podotknout, že dělení textu do hřebů je časově náročná činnost, jejíž složitost narůstá s délkou analyzovaného textu. Software, který vznikl v rámci této diplomové práce, obsahuje snadno ovladatelné uživatelské rozhraní, které dělení textu do hřebů co možná nejvíce usnadňuje.

Pro první strofu Slavíkova překladu můžeme etablovat následující hřeby:

\begin {table}[H]
	\caption {Hřeby etablované pro první strofu překladu Ivana Slavíka} 
	\label{tab:hreby-1-strofa-slavik} 

	\begin{center}
		\begin{tabular}{|p{10cm}|c|c|}
		\hline 
		\bfseries Hřeb & \bfseries Rozsah & \bfseries Počet \\ \hline
		\textbf{6 já} [(hloub)al jsem 8, (lušti)l 12, (nedříma)l jsem 20, (nebdě)l 22, mě 37], \textbf{25 zaklepání} [(ozva)lo se 30, zaklepání 31, zaklepání 32, (chc)e 42, zaklepání 44] 
		& 5 & 2 \\ \hline
		
		\textbf{1 pusto} [(byl)o 2, pusto 3], \textbf{2 být} [bylo 1, je 46], \textbf{5 když} [když 6, když 24], \textbf{27 a} [a 33, a 50], \textbf{28 víc} [víc 34, víc 52], \textbf{29 ne} [ne 35, ne 53], \textbf{30 kdo} [kdo 36, (shán)í 39], \textbf{18 kajícný} [kajícné 18, kajícné 45], \textbf{23 ťuk} [ťuk 26, ťuk 27], \textbf{34 to} [toto 43, to 48], \textbf{35 host} [(je) 47, host 49] 
		& 2 & 11\\ \hline
		
		\textbf{3 o} [o 4], \textbf{4 půlnoc} [půlnoci 5], \textbf{7 hloubat} [hloubal jsem 7], \textbf{8 bez} [bez 9], \textbf{9 pomoc} [pomoci 10], \textbf{10 luštit} [luštil 11], \textbf{11 marně} [marně 13], \textbf{14 živou} [živou 14], \textbf{15 moc} [mocí 15], \textbf{16 starý} [staré 16], \textbf{17 svazek} [svazky 17], \textbf{19 dřímat} [nedřímal jsem 19], \textbf{20 bdít} [nebděl 21], \textbf{21 ani} [ani 23], \textbf{22 tu} [tu 25], \textbf{24 znenadání} [znenadání 28], \textbf{26 ozývat} [ozvalo se 29], \textbf{31 shánět} [shání 38], \textbf{32 co} [co 40], \textbf{33 chtít} [chce 41], \textbf{36 nic} [nic 51]
		& 1 & 21 \\ \hline

		\end{tabular} 
	\end{center}
\end{table}
\subsection{Další typy hřebů}
Hřeby, které vznikly analýzou textu, je možné transformovat na
\begin{enumerate}
\item \textbf{Množinový hřeb}\\
Množinový hřeb (značen $\{\kern 2pt.\kern 2pt\}$) je takový hřeb, v němž se uvedou pouze unikátní denotační elementy. Vyskytují-li se v hřebu denotační elementy související s deklinací, jsou seskupeny do kategorie. Například pro hřeb číslo $25$ bychom vytvořili množinový hřeb

$\text{zaklepání} = \{\kern 2pt \text{zaklepání, 3. osoba singuláru} \kern 2pt\}$.

\item \textbf{Poziční hřeb}\\
Je značen $\langle \kern 2pt.\kern 2pt\rangle$. V pozičním hřebu jsou denotační elementy znázorněny pouze svým pořadovým číslem. Poziční hřeb šestého hřebu \enquote{já} můžeme určit jako

$\text{já} = \langle \kern 2pt \text{8, 12, 20, 22, 37} \kern 2pt\rangle$.
\end{enumerate}

\section{Charakteristiky}

Texty zachycují určitou skutečnost, kterou čtenář při jejich čtení \uv{rekonstruuje}. Skutečnost je v~textu zachycena formou pojmů. Aby čtenář text pochopil v souladu s úvahami autora, je třeba, aby byl v textu zachycen vztah mezi různými pojmy.

Denotační analýza zkoumá text právě z tohoto hlediska --- sleduje, jak spolu jednotlivé v~textu se vyskytující elementy souvisí. Při denotační analýze mohou být určovány následující charakteristiky:

\subsection{Jádro textu}
Jádro textu obsahuje ty hřeby, v nichž se vyskytují nejméně dva odlišné kmeny\footnote{Kmenem se v lingvistice označuje ta část slova, která se v různých tvarech téhož slova (u flektivních slovních druhů) nemění. Mnohdy bývá totožná s kořenem (morfémem nesoucím základní významovou (lexikální) část slova. \url{http://cs.wikipedia.org/wiki/Kmen_(mluvnice)} }. Do jádra textu první strofy Slavíkova překladu tedy patří hřeby \uv{1 pusto}, \uv{6 já}, \uv{25 zaklepání}, \uv{30 kdo}, \uv{35 host}.

\subsection{Velikost hřebu}
Velikost (též rozsah) hřebu je definována jako počet elementů v daném hřebu. Velikost hřebů vytvořených z textu první strofy překladu Ivana Slavíka je uvedena ve druhém sloupci tabulky \ref{tab:hreby-1-strofa-slavik}.

\subsection{Kardinální číslo jádra}
Kardinální číslo jádra se značí $|\kern 2pt.\kern 2pt|$. Je dáno součtem velikostí hřebů, které jsou součástí jádra textu. Pro analyzovanou strofu platí
\begin{equation}
	|\kern 2pt.\kern 2pt| = 2 + 5 + 5 + 2 + 2 = 16.
\end{equation}
\subsection{\uv{Lokálnost} jádrových hřebů}
\uv{Lokálnost} jádrových hřebů (značena $T$) zobrazuje dominantnost určitých hřebů v textu.  Jedná se o poměr velikosti hřebu ke kardiálnímu číslo jádra:
\begin{equation}
T_\text{(hřeb)}=\frac{\text{velikost hřebu}}{|\kern 5pt.\kern 5pt|}.
\end{equation}
Viz \cite[str. 302]{Wimmer2003}. 

Hodnota lokálnosti jádrových hřebů pro jádrové hřeby \uv{6 já} a \uv{25 zaklepání} je $T_\text{(já)} = T_\text{(zaklepání)} = 5 / 16 = 0{,}3125$, pro zbylé jádrové hřeby $T_\text{(pusto)} = T_\text{(kdo)} = T_\text{(host)} = 2 / 16 = 0{,}125$.

\subsection{Deterministické jádro}

Deterministické jádro souvisí s problematikou  \glslink{tema_rema}{tématu a rématu}. Do deterministického jádra patří všechny hřeby, které na sebe odkazují. Například vidíme, že se ve hřebu \uv{$25$ zaklepání} [(ozva)lo se 30, zaklepání 31, zaklepání 32, (chc)e 42, zaklepání 44] nachází denotační element [(ozva)lo se 30]. V deterministickém jádru proto bude obsažen jak hřeb \uv{$25$ zaklepání}, tak hřeb \uv{$26$ ozývat}, který obsahuje element [ozvalo se 29].


\subsection{Kompaktnost textu}
Kompaktnost textu určuje, jak moc je text zaměřen na jeden (hlavní) motiv. To přímo úměrně souvisí s počtem hřebů, které se v textu vyskytují. Kompaktnost textu $K$ je číslo ležící v intervalu $\langle0, 1\rangle$. Čím je vyšší, tím je text kompaktnější. Naopak čím nižší je hodnota $K$, tím více je text rozvětven. Vzorec pro výpočet hodnoty $K$ je v \cite[str. 303]{Wimmer2003} definován jako
\begin{equation}
K=\frac{1-\frac{N}{L}}{1-\frac{1}{L}},
\end{equation}
kde 
\begin{align*}
      N & \text{ je počet v textu nalezených hřebů,}\\
      L & \text{ je počet denotačních pozic v textu.}
\end{align*}     

Pro první strofu Slavíkova překladu bychom tedy dostali

\begin{equation}
K=\frac{1-\frac{N}{L}}{1-\frac{1}{L}} = \frac{1-\frac{34}{53}}{1-\frac{1}{53}}=0{,}365.
\end{equation}

\subsection{Centralizovanost textu}
Centralizovanost textu ($R$) vyjadřuje sílu, jakou se denotace koncentruje na centrální hřeby. Je závislá na počtu výskytů jednotlivých hřebů. Spočítá se (viz \cite[str. 303]{Wimmer2003}) jako
\begin{equation}
R=\frac{1}{n^2}\sum{ \abs{H_i}^2 \times f_i},
\end{equation}
kde 
\begin{align*}
      \abs{H_i} & \text{ je velikost hřebu,}\\
      f_i & \text{ je počet hřebů s velikostí \abs{H_i},}\\
      n&=\sum{ \abs{H_i}\times f_i }.
\end{align*}     

Pomocí MacIntoshova indexu je možné spočítat relativní centralizovanost textu. Vzorec výpočtu MacIntoshova indexu je definován v \cite[str. 304]{Wimmer2003} jako
\begin{equation}
R_{rel}=\frac{1-\sqrt{R}}{1-\frac{1}{\sqrt{n}}}.
\end{equation}

Pro výpočet centralizovanosti textu na základě hřebů uvedených v tabulce \ref{tab:hreby-1-strofa-slavik} dostaneme:
\begin{equation}
\begin{aligned}
\sum{ \abs{H_i}^2 \times f_i} &= 5^2 \times 2 + 2^2 \times 11 + 1^2 \times 21 = 115,\\
n &=  5 \times 2 + 2 \times 11 + 1 \times 21 = 53,\\
n^2 &= 2756,\\
R &=\frac{115}{2756} = 0{,}042,\\
R_{rel} &= \frac{1-\sqrt{0{,}042}}{1-\frac{1}{\sqrt{53}}} = 0{,}92.
\end{aligned}
\end{equation}
\subsection{Difuze hřebu}
Jedná se o relativní rozdíl mezi prvním a posledním výskytem slova z jednoho hřebu v textu. Z toho vyplývá, že počítat difuze má smysl pouze pro hřeby o velikosti alespoň $2$.

Vztah pro výpočet difuze hřebu ($D_{\text{hřeb}}$) je dána jako:
\begin{equation}
D_\text{hřeb}=\frac{\max{\langle\text{hřeb}\rangle}-\min{\langle\text{hřeb}\rangle}}{\text{velikost hřebu}},
\end{equation}
viz \cite[str. 304]{Wimmer2003}. Difuze hřebu \uv{6 já} se tedy vypočítá jako

\begin{equation}
D_\text{já}=\frac{37 - 8}{5} = 5{,}8.
\end{equation}\textbf{}


\subsection{Rozšířené jádro textu}
Rozšířené jádro je charakteristické pro difuzi textu. Patří do něj všechny hřeby, jejichž difuze je menší nebo rovna nejvyšší difuzi jádrového hřebu, přičemž jsou vynechány všechny předložky, spojky, zájmena a citoslovce.

\subsection{Koincidence}
Koincidence je definována jako výskyt dvou hřebů v jedné vymezené části textu (verš, věta, sloka, \ldots). Koincidence je považována za:
\begin{description}
\item[náhodnou,] pokud pravděpodobnost výskytu daných dvou hřebu je dostatečně vysoká,
\item[signifikantní,] v případě, že pravděpodobnost tohoto výskytu je malá.
\end{description}

Hranice mezi \textit{náhodnou} a \textit{signifikantní} koincidencí je označena $\alpha $. Je-li vypočítaná pravděpodobnost dané koincidence menší nebo rovna $\alpha $, je tato koincidence nazývána \textit{signifikantní koincidence na hladině $\alpha $}. Hodnotu $\alpha $ je třeba volit v závislosti na typech textu a velikosti vymezené části textu, v níž je koincidence zkoumána.

Pro výpočet koincidence je třeba upravit zkoumaný text, a to tak, že se slova v textu nahradí pořadovým číslem hřebu. Pro každou dvojici hřebů $A$ a $B$ se poté pomocí vzorce
\begin{equation}
P(X \geq x)=\sum_{i=x}^{min{\left(  M, n \right) }} \frac{\binom{M}{i}\binom{N-M}{n-i}}{\binom{N}{n}},
\end{equation}
kde 
\begin{align*}
	N & \text{ je počet vymezených částí textu (počet veršů, vět, slok, \ldots),}\\
	M & \text{ je počet vymezených částí, v nichž se vyskytuje hřeb $A$,}\\
	n & \text{ je počet vymezených částí, v nichž se vyskytuje hřeb $B$,}\\
	x & \text{ je počet vymezených částí, v nichž se vyskytuje jak hřeb  $A$, tak hřeb $B$,}
\end{align*}  

vypočítá kumulativní pravděpodobnost hypergeometrického rozdělení (viz \cite[str. 307]{Wimmer2003}). Koincidence je \textit{signifikantní}, pokud platí $P(X \geq x)\leq \alpha$. Čím je zvolená hladina $\alpha$ menší číslo, tím méně hřebů je vzájemně \textit{signifikantních}.

Z vypočítaných pravděpodobností $P(X \geq x)$ je možné sestrojit $\alpha$-graf. V $\alpha$-grafu jsou hřeby zakresleny jako vrcholy, přičemž sousedními se stanou ty vrcholy (=hřeby), jejichž vzájemná pravděpodobnost $P(X \geq x)$ je menší nebo rovna $\alpha$ (tzn., jsou \textit{signifikantní}).

Z toho vyplývá, že volba velikosti hladiny $\alpha$ má vliv na počet hran a komponent ve výsledném $\alpha$-grafu. 

Charakteristikami, které jsou spojeny s koincidencí, jsou:
\begin{description}
	\item[Index nespojitosti] představuje nejsilnější vztah v textu. Je-li hladina $\alpha$ stanovena na hodnotu nižší, než je \textit{index nespojitosti}, bude $\alpha$-graf tvořen pouze izolovanými vrcholy.
	\item[Index neizolovanosti] je pravděpodobnost zaručující, že žádný z vrcholů v $\alpha$-grafu nebude izolovaný. Pokud je hladina $\alpha$ stanovena nad \textit{index neizolovanosti}, každý vrchol $\alpha$-grafu bude incidentní s alespoň jednou hranou.
	\item[Index dosažitelnosti] je taková hodnota $\alpha$, při které je celý $\alpha$-graf tvořen pouze jednou komponentou. To znamená, že $\alpha$-graf je souvislý a tudíž existuje cesta mezi jeho libovolnými dvěma vrcholy.
\end{description}

\subsection{Souvislost}
V teorii grafů je za souvislý považován takový graf, ve kterém pro každé jeho dva vrcholy $x$ a $y$ existuje v $G$ cesta z $x$ do $y$. Souvislost grafu vyjadřuje, je-li možné dostat se z každého vrcholu do jiného, nebo zda jde o více na sebe nenavazujících komponent.

Souvislost charakterizuje, jak hustá je síť hran v grafu. Z lingvistického hlediska naznačuje míru koncentrace zachycené reality. Ziegler a Altman používají v \cite{ZieglerAltmann2002} dvě míry souvislosti:

\textit{Relativní míra souvislosti}, která se vypočítá jako
\begin{equation}
\kappa_\textit{rel}^\alpha=\frac{n-k}{n-1},
\end{equation}

kde
\begin{align*}
	n & \text{ je počet hřebů,}\\
	k & \text{ je počet komponent grafu.}
\end{align*}  
\textit{Relativní cyklomatické číslo}, které bere v potaz také počet hran $m$. Vypočítá se jako
\begin{equation}
\mu_\textit{rel}^\alpha=\frac{2(m-n+k)}{(n-1)(n-2)}.
\end{equation}

\subsection{Stupeň vrcholu}
Stupeň vrcholu je definován jako počet hran, které z vrcholu vycházejí. V případě koincidenčních grafů mívají jádrové hřeby poměrně nízké stupně. To je dáno tím, že s ostatními (nejádrovými) hřeby koincidují většinou náhodně.

Pro získání grafu s lépe odpovídajícími asociacemi, je třeba zkombinovat graf koincidenční s grafem deterministickým. V tom případě jsou hranou propojeny i vrcholy, které reprezentují vzájemně na sebe odkazující hřeby. Tyto odkazy jsou v textu dané, proto jsou označeny jako deterministické.

Text je možné charakterizovat mírou konotativní koncentrace. Jedná se o poměr mezi pozorovaným a možným počtem hran (\cite[str. 312]{Wimmer2003}):
\begin{equation}
d_\textit{rel}=\frac{2m}{n(n-1)}.
\end{equation}

\subsection{Vzdálenost v grafu}
Vzdálenost mezi dvěma vrcholy $x$ a $y$ (označena jako $d(x,y)$) je v teorii grafů vyjádřena jako nejmenší počet hran, po nichž je možné přejít z $x$ do $y$. Nejsou-li vrcholy $x$ a $y$ vůbec spojeny, je $d(x,y)=\infty$.

V jednotlivých komponentách grafu je možné vypočítat vzdálenosti mezi všemi dvojicemi vrcholů v dané komponentě. Největší vzdálenost, kterou je možné z  vrcholu $x$ dosáhnout v~rámci jeho komponenty, je nazývána \textit{excentricita vrcholu}. Je definována na straně 313 v \cite{Wimmer2003} jako
\begin{equation}
e(x)=\max_{y \in K} d(x,y).
\end{equation}

Diametrem komponenty $K$ nazýváme nejvyšší excentricitu vrcholu v rámci dané komponenty:
\begin{equation}
dm(G)=\max_{x \in K} e(x),
\end{equation}

viz \cite[str. 315]{Wimmer2003}.

Diametr komponenty vyjadřuje denotační šířku --- šířku úseku reality, kterou autor konstruoval jako souvislý celek. Vrcholy s nejmenší excentricitou jsou nazývány centrem komponenty. V jedné komponentě se může nacházet více center.
Centrální index udává sílu asociace mezi hřeby. Je dán v \cite[str. 315]{Wimmer2003} jako
\begin{equation}
\bar{d}=\frac{z(K)}{n_K (n_K-1)},
\end{equation}

kde
\begin{align*}
	z(K) & \text{ je součet všech vzdáleností v dané komponentě: 
$z(K)=\sum_{(x,y \in K)} d(x,y)$,}\\
	n_k & \text{ je počet vrcholů v komponentě $K$.}\\
\end{align*} 

Nevýhodou centrálního indexu je, že asociativita roste rychleji, než přibývají nové hřeby. Staré hřeby nabývají více asociací a vytváření nových hřebů je stále méně časté, což vede ke zkracování vzdáleností.


\subsection{Relativní míra centrality}
Relativní míru centrality definují Ziegler a Altman v \cite[str. 315]{ZieglerAltmann2002}. Její výhodou je eliminování výše popsaného nedostatku centrálního indexu. Je definována jako
\begin{equation}
Z(K)=\frac{z_{max}-z(K)}{z_{max}-z_{min}}=
\frac{(n_K+1) n_K (n_K-1)-3z(K)}{n_K (n_K-1)(n_K-2)} .
\end{equation}


\end{document}
