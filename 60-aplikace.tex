\documentclass[dp.tex]{subfiles}
\begin{document}
\chapter{Aplikace}

V rámci této diplomové práce byla vytvořena aplikace umožňující uživateli snadné provádění dříve zmíněných testů a analýz. Aplikace je tvořena dvěma nezávislými moduly:
\begin{itemize}
\item knihovnou \texttt{compLing},
\item modulem obsahujícím grafické uživatelské rozhraní -- GUI.
\end{itemize}

Oba moduly jsou naprogramovány v programovacím jazyce Java. Pro spuštění aplikace je proto potřeba nejprve nainstalovat běhové prostředí Java Virtual Machine \footnote{Dostupné \url {https://java.com/en/download}}. Součástí přílohy této diplomové práce je CD, na kterém je mimo jiné i spustitelná verze aplikace ve formátu JAR.

Velkou výhodu Javy, jak napovídá i její motto \enq{write once, run anywhere}, je její multiplatformnost. To znamená, že v Javě napsaná aplikace bude bez úprav spustitelná na všech platformách, pro něž je dostupná Java Virtual Machine. 

\section{Architektura aplikace}

Na začátku vývoje bylo třeba zvolit architekturu budoucí aplikace. Po předešlých zkušenostech z oblasti vývoje softwaru jsem se rozhodl pro dva relativně nezávislé moduly, neboť čím je program jednodušší, tím méně je náchylný k chybám. 

Prvním z těchto modulů měla být knihovna, která bude provádět samotnou analýzu textu. Tato knihovna musela být z důvodu co nejvyšší spolehlivosti důkladně otestována. Dalšími požadavky byly:
\begin{itemize}
\item rychlost,
\item multiplatformnost,
\item snadná rozšiřitelnost (např. v rámci jiné závěrečné práce).
\end{itemize}

Druhý modul měl tuto knihovnu využívat a poskytnout k ní uživatelsky přívětivé grafické rozhraní (tzv. \acrshort{gui}). Grafické rozhraní mělo umožnit uživateli otevřít textový soubor (případně více souborů), provést na něm za pomoci modulu knihovny požadovanou analýzu a zobrazit její výsledky. Důležitým požadavkem na modul grafického rozhraní byla možnost exportu vstupních dat do souboru \acrshort{csv}. 

V rámci tohoto modulu pro mě zůstala stěžejní multiplatformnost a snadná rozšiřitelnost. Vzhledem k tomu, že cílem bylo vytvořit jednoduchý frontend, nepředpokládal jsem, že by aplikace mohla být pomalá. Od automatizovaného testování jsem upustil, počítal jsem pouze s testováním uživatelským.



\section{Knihovna \texttt{compLing}}

\section{Grafické uživatelské rozhraní}

\end{document}