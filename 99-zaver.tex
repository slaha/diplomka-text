\documentclass[dp.tex]{subfiles}

\begin{document}
\chapter*{Závěr}
\addcontentsline{toc}{chapter}{Závěr} 
\label{chap:zaver} 

Cílem diplomové práce bylo prostudování vybraných pojmů a prostředků matematické lingvistiky a jejich aplikování na českých překladech básně Havran. Součástí práce bylo také \mbox{vytvoření} počítačového programu, který provádí automatický výpočet vybraných lingvistických charakteristik a umožňuje porovnávání dosažených výsledků.

Tato práce navazuje na mou bakalářskou práci taktéž věnovanou lingvistice. V rámci bakalářské práce byl vytvořen program umožňující provádění analýz vybraných jevů, kterými byly frekvence písmen a slov a fonické jevy asonance, aliterace a agregace. Tento program byl využit při tvorbě článku \enquote{Měření podobnosti překladů básně Havran}, kdy se projevily jeho slabiny. Těmi byla jednak nepřesnost některých analýz a jednak nemožnost exportu naměřených hodnot pro zpracování externím nástrojem.

Kromě odstranění zmíněných nedostatků si tato práce kladla za cíl rozšíření podporovaných typů analýz o denotační analýzu. Největším rozdílem mezi programy vytvořenými v~rámci bakalářské a diplomové práce je jejich architektura. Zatímco program vyvinutý v~rámci bakalářské práce byl monolit obsahující veškerou funkcionalitu, v diplomové práci je aplikace rozdělena do dvou modulů. První z modulů obsahuje funkcionalitu týkající se pouze analýzy textů a sám neposkytuje žádné jiné rozhraní než aplikační. Aby s tímto modulem mohl pracovat uživatel, musí se stát modul součástí aplikace, která k němu poskytne i rozhraní \mbox{uživatelské}. To poskytuje druhý ze zmiňovaných modulů.

Vytvořená aplikace je pro potřeby analýzy textů dobře použitelná. Unikátní je zejména část zabývající se denotační analýzou. Bylo by nicméně odvážné tvrdit, že aplikaci nelze dále \mbox{vylepšovat}. Rezervy vidím zejména ve zdrojových kódech aplikace, které by zasloužily tzv.~refaktorovat --- upravit strukturu zdrojového kódu bez toho, aby byla ovlivněna jeho funkčnost. Díky rozdělení do dvou modulů je aplikace dobře připravena na rozšiřování funkčnosti o další typy analýz.

V rámci práce byly provedeny analýzy osmnácti českých překladů básně Havran a jejich vzájemné porovnání. Testovány byly všechny podporované typy analýz. Některé analýzy byly pro srovnání provedeny i pro původní anglický text básně The~Raven. Pro denotační analýzu byl uvažován pouze text $14.$~strofy, a to zejména kvůli časové náročnosti vytvoření hřebů pro celou báseň.

Srovnání četností písmen bylo provedeno na všech osmnácti testovaných českých překladech. Ty obsahují celkem $69368$ znaků, z nichž $40$~bylo unikátních. Nejčastěji se vyskytujícím znakem byl s $5530$ výskyty znak \textit{E}. Následovaly znaky \textit{N}, \textit{A}, \textit{O} a s větším odstupem znak~\textit{S}. Naměřená relativní četnost byla srovnána s frekvenčním seznam dostupným na stránkách czech-language.cz\footnote{Dostupné on-line z \url{http://www.czech-language.cz/alphabet/alph-prehled.html}}, který vznikl analýzou souboru o velikosti $3 139 926$ slov. V něm se pořadí na prvních pěti místech liší --- nejfrekventovanějším znakem je \textit{O}, dále \textit{E}, \textit{N}, \textit{A} a \textit{T}. Největší rozdíl mezi relativní frekvencí znaků v analyzovaných překladech a použitým frekvenčním seznamem bylo dosaženo u písmene \textit{O}, a to $2{,}32\%$.

V rámci analýzy frekvence slov byla provedena dvě porovnání. V prvním byla srovnávána frekvence slov v českých překladech s frekvencí uvedenou ve frekvenčních slovnících, ve druhém byla srovnána frekvence výskytů slov \enquote{havran} a \enquote{pták} mezi českými překlady a anglickým originálem. 

Osmnáct českých překladů obsahuje celkem $16091$ slov, z toho je $3938$ slov unikátních. Nejčastěji se vyskytujícím slovem v českých překladech je slovo \enquote{v}, které se na předních příčkách umístilo i ve frekvenčních slovnících použitých pro srovnání. Dalšími často se vyskytujícími slovy byla slova \enquote{se}, sloveso \enquote{být}, spojka \enquote{a}, předložka \enquote{na}. V první desítce se objevila i dvě slova, kterým jinak ve frekvenčních slovnících patří nižší příčky --- slova \enquote{havran} a \enquote{pták}. To je samozřejmě dáno obsahem analyzovaných básní. Další odlišností naměřených a~referenčních dat je relativní frekvence výskytu jednotlivých slov. Ta je v případě českých překladů obecně vyšší než ve frekvenčních slovnících, což je dáno malou velikostí a variabilitou testovaných textů.

V originálním textu E. A. Poea jsou slovo \enquote{havran} a \enquote{pták} (respektive \enquote{raven} a \enquote{bird}) zastoupena shodně desetkrát. Z českých překladů dosahuje pouze překlad Jiřího Taufera těchto četností. V překladu Jaroslava Vrchlického z roku 1881 se \enquote{havran} vyskytuje stejně jako v~originálu $10\times$, četnost výskytu slova \enquote{pták} je však nižší --- devět. Překlady Rudolfa Havla a Augustina Eugena Mužíka obsahují shodně jedenáct výskytů slova \enquote{havran} a devět výskytů slova \enquote{pták}. Nejvíce od originálu vzdálený překlad v rámci této analýzy vyšel překlad Vratislava Kazimíra Šembery, v němž se \enquote{havran} vyskytl celkem $16\times$, naproti tomu \enquote{pták} pouze třikrát.

Analýza agregace byla provedena na českých překladech, přičemž pro každý překlad bylo možné s rostoucí vzdáleností veršů pozorovat jejich klesající podobnost. Od tohoto trendu se však vždy vyskytla výjimka v závislosti na tom, zda byl analyzovaný překlad strukturován do strof o šesti nebo jedenácti verších. V případě překladů, jejichž strofy jsou tvořeny šesti verši, vykazovaly odchylku hodnoty naměřené pro vzdálenost $6$. Překlady dělené do strof tvořených jedenácti verši se lišily pro vzdálenost $2$.

Při analýze asonance byla testována pouze první ze dvou možných hypotéz $H_0$ zkoumající, zda jsou střední hodnoty vzhledem k různým posunům stejné. Analýzou bylo zjištěno, že střední hodnoty pro různé posuny stejné nejsou. Lišilo se celkem šestnáct dvojic posunů. Jako nejméně podobný ostatním vyšel posun o $8$, který se vyskytl v devíti ze šestnácti lišících se dvojic.

Aliterace byla provedena jak pro originální anglický text, tak pro české překlady. Hodnota $\alpha$ byla zvolena jako $0{,}05$. Aliterační charakter \KA originálního textu vyšel $3{,}79759$. České překlady je možné rozdělit na dvě skupiny, které odpovídají struktuře básně. Pro překlady strukturované do strof o jedenácti verších se hodnoty \KA pohybovaly v intervalu přibližně $( 1{,}5;2 )$, zatímco aliterační charakter překladů, jejichž strofy ctí strukturu originálu a jsou tvořeny šesti verši, dosahoval hodnot přibližně z intervalu $\left( 3; 4 \right)$. Aliteračnímu charakteru originálu byl nejblíže překlad Eugena Stoklase s ${K\!A} = 3{,}87856$.

Při denotační analýze byly etablovány hřeby pro $14.$ strofu původního  anglického textu i jeho českých překladů. K omezení analyzované básně na pouze jednu strofu bylo přistoupeno z důvodu časové náročnosti, neboť složitost dělení textu do hřebů stoupá spolu s~délkou textu. Na základě etablovaných hřebů byly vypočítány indexy nespojitosti, neizolovanosti a dosažitelnosti, které mohou být považovány za charakteristiky textu. Index dosažitelnosti dosáhl u originálu i většiny překladů hodnoty $\infty$, což je způsobeno posledním veršem, který se na rozdíl od zbytku strofy věnuje havranovi. Z devatenácti textů je pouze pět charakterizováno unikátními indexy. Zbývajících čtrnáct textů je dle hodnot jednotlivých indexů rozděleno do pěti skupin, přičemž o textech v jedné skupině můžeme tvrdit, že jsou si podobné. Nejpodobnější originálnímu překladu se z tohoto pohledu jeví překlady Karla Dostála-Lutinova, Rudolfa Havla, Rudolfa Černého a Ivana Slavíka.
\end{document}
