\documentclass[dp.tex]{subfiles}


\begin{document}
\chapter*{Úvod}
\addcontentsline{toc}{chapter}{Úvod} 
\label{chap:uvod} 


Tato diplomová práce tematicky navazuje na moji bakalářskou práci \citetitle{Slahora2012}\footnote{\cite{Slahora2012}: \fullcite{Slahora2012}}. Cílem práce bylo odpovědět na otázku \enquote{Lze pomocí fonických jevů rozlišit autorství textu, respektive jazyk?} V rámci bakalářské práce bylo autorství (resp. jazyk) zkoumáno na deseti básních deseti českých autorů. Použitá metodika využívala k určení autorství (jazyka) různé charakteristiky jako průměrnou délku rýmu, četnosti slov různých délek a fonické jevy: agregaci, asonanci a aliteraci.

Součástí práce byl i počítačový program, který počítal dané charakteristiky a na jejich základě texty porovnával. Tento program byl následně využit při psaní článku \citetitle{Marek2013}\footnote{\cite{Marek2013}: \fullcite{Marek2013}}. Při tvorbě článku se však projevily nedostatky, které původní program obsahoval. Jednak se jednalo o nemožnost exportu naměřených charakteristik pro další zpracování jinými specializovanějšími nástroji a jednak výsledky analýz trpěly nepřesnostmi, které nebyly objeveny v rámci bakalářské práce.

Diplomová práce si klade za cíl rozšíření možností textových analýz o \textit{denotační analýzu} a porovnání sledovaných charakteristik u různých českých překladů básně Havran amerického autora Edgara Allana Poea.

\end{document}
