% !TeX root = 70-experiment.tex
\documentclass[dp.tex]{subfiles}
\begin{document}
\chapter{Analýza}
\label{chap:experiment} 

Cílem diplomové práce je porovnání sledovaných charakteristik na různých českých překladech básně Havran. České překlady byly vydány nakladatelstvím Odeon v knize \citetitle{Poe1990}. Pro realizaci praktické části této diplomové práce budou použity překlady uvedené v knize, a to včetně dvou překladů (Vrchlického překlad vydaný roku 1881 v časopisu Lumír a čtvrté vydání překladu Kamilla Reslera vydaném Josefem Ciprou roku 1956) uvedených v komentáři Rudolfa Havla.

\section{Porovnání četností znaků}

Prvním porovnáním, které bylo v rámci praktické části realizováno, je srovnání četnosti znaků. Jak uvádí Jan Králík v \cite[str.~109]{Tesitelova1987}: 

\begin{quote}
Znalost míry užívání (frekvence) písmen stála u zrodu těsnopisných a telegrafních značek a dodnes tak úzce souvisí s ekonomikou záznamu a přenosu textových informací. Znalost frekvence písmen výrazně zasáhla i do uspořádání klávesnic u prvních psacích strojů sériově vyráběných; dodnes jsou proto psací stroje s latinkou vhodnější pro angličtinu, pro niž byly původně konstruovány, než pro kterýkoli jiný jazyk. Znalost frekvencí písmen se osvědčila i ve vojenství při dešifrování kódovaných zpráv: stačilo znát frekvence písmen v daném jazyce a porovnat je s frekvencemi užitých \clq tajných znaků\crq.
\end{quote}

Porovnáváno bylo všech osmnáct českých překladů, přičemž analýza byla nastavena tak, aby byla započítána pouze písmena, aby byla ignorována rozdílná velikost písmen a aby \glslink{bigram}{bigram} \clq ch\crq byl považován za jedno písmeno.

V takto nastavené analýze četnosti znaků bylo v testovaných textech nalezeno $69 368$ znaků, z toho však pouze $40$ znaků bylo unikátních (z české abecedy nebyly zastoupeny znaky \textit{Q} a \textit{W}). Nejčastěji se vyskytujícím znakem byl znak \textit{E}, který se vyskytl celkem $5530\times$. Následovaly znaky \textit{N}, \textit{A}, \textit{O}, a \textit{S}.

Námi nalezené četnosti porovnáme s četnostmi uvedenými na stránkách czech-language.cz\footnote{Dostupné on-line z \url{http://www.czech-language.cz/alphabet/alph-prehled.html}}. Tato frekvenční tabulka byla vytvořena na základě analýzy $3 139 926$ znaků. Na Internetu i v literatuře je možné najít i další frekvenční tabulky pro češtinu\footnote{on-line např. \url{https://nlp.fi.muni.cz/web3/cs/FrekvenceSlovLemmat}, \url{http://www.sttmedia.com/characterfrequency-czech}, \url{http://www.cryptogram.org/cdb/words/frequency.html}, v literatuře např. \citetitle{Kraus1965} (viz \cite{Kraus1965}), částečně \citetitle{Tesitelova1987} \cite[str.~109-121]{Tesitelova1987}.}.

\begin{figure}[h!]
	\centering
	\includegraphics[max width=\textwidth,keepaspectratio=true]{imgs-70-prakticka/cetnost-znaku2}
	\caption[Srovnání relativních četností výskytu znaků v analyzovaných překladech.]{Srovnání relativních četností výskytu znaků v analyzovaných překladech. \textit{Zdroj:~vlastní.}}
	\label{fig:character-freq}
\end{figure}

Graf \ref{fig:character-freq} zachycuje poměr mezi relativními četnostmi výskytu jednotlivých znaků v analyzovaných překladech. V grafu vidíme, že naměřené hodnoty korespondují s hodnotami z~referenční frekvenční tabulky. Nejvyšší odchylka se vyskytla u znaku \textit{O}, činí $2{,}32 \%$.

\section{Porovnání četností slov}

Zkoumání počtu výskytů jednotlivých slov v daném jazyce je typickou úlohou kvantitativní lingvistiky. Její výsledky mohou být použity při konstruování slovníků a jazykových učebnic, k identifikaci klíčových slov v textech, k poměřování mezi alternujícími slovy (např. \enquote{bychom} a \enquote{bysme}), k nalezení ustálených slovních spojení aj.\footnote{Zdroj: \url{https://wiki.korpus.cz/doku.php/pojmy:frekvence\#vyuziti_a_vyznam_frekvence}.}.

\sloppy
To, kolikrát se které slovo vyskytuje, uvádějí tzv. frekvenční slovníky nebo seznamy. Pro češtinu existuje několik takových slovníků (\cite[str.~18]{Tesitelova1987}), v literatuře např. \cite{Jelinek1961}, \cite{Tesitelova1980}, on-line dostupné např. \url{https://ucnk.ff.cuni.cz/srovnani10.php}. Výsledky analýzy četnosti znaků budou porovnány s následujícími frekvenčními seznamy:

\begin {table}[H]
	\caption {Srovnání frekvenčních slovníků} \label{tab:title} 

	\begin{center}
		\begin{tabular}{{l|l|l|l}}
		\hline
		\bfseries \bfseries Název & \bfseries Rok publikace & \bfseries Zdroj & \bfseries Velikost zdrojového souboru [slova]\\
		    \hline \hline
		   Jelínek     & 1961 & \cite{Jelinek1961}    &    1 623 527   \\\hline
		   Těšitelová  & 1983 & \cite{Tesitelova1980} &     540 000    \\\hline
		   Čermák      & 2004 & \cite{Cermak2004}     & 100 000 000    \\\hline
		\end{tabular}
	\end{center}
	\label{tab:word-freq-lists}
\end{table}

V rámci ukázky analýzy četnosti slov budou předvedena dvě porovnání:
\begin{enumerate}
\item V českých překladech bude sledován počet slov se zaměřením na slova odkazující na \enquote{havrana} (tj. slova \enquote{havran} a \enquote{pták} + jejich tvary).
\item České překlady budou porovnány s anglickým originálem. Porovnáván bude počet odkazů na havrana v originálu (anglická slova \enquote{raven} a \enquote{bird}) s českým \enquote{havran} a \enquote{pták} v jednotlivých překladech.
\end{enumerate}

\subsection{Analýza českých překladů}
\label{chap:word-freq-list} 

Analýza četnosti slov byla nastavena tak, aby byla slova porovnávána dle obsahu a aby byla ignorována velikost písmen. Vzhledem k tomu, že vyvinutý program nepodporuje žádný slovník tvarů slov, bylo nutné přidat do analýzy taková pravidla, aby byly i výskyty různých tvarů slov \enquote{havran} a \enquote{pták} připočteny k výskytům \glslink{lemma}{lemmat}. Pro slovo \enquote{havran} se jednalo o tvary \textit{havrana}, \textit{havrane}, \textit{havráně} a \textit{havrany}. Tvary slova \enquote{pták}, které se nachází v jednotlivých překladech, jsou: \textit{ptáka}, \textit{ptákem}, \textit{ptákovi}, \textit{ptáku} a \textit{ptáků}.

Osmnáct testovaných překladů obsahuje celkem $16 091$ slov, z nichž bylo $3938$ slov unikátních. Nejčastěji se vyskytujícím slovem bylo slovo \enquote{v} ($425$ výskytů) následované slovy \enquote{se} (nalezeno $349\times$) a \enquote{jsem} ($331$ výskytů). \enquote{Havran} se se $173$ výskyty umístil na $7.$ místě, \enquote{pták} na místě $9.$, když byl nalezen $149\times$.

Na obrázku \ref{fig:word-freq} je vyobrazeno porovnání relativních četností výskytu deseti nejčastějších slov v analyzovaných překladech s četnostmi pocházejícími z frekvenčních seznamů uvedených v tabulce \ref{tab:word-freq-lists}.

\begin{figure}[h!]
	\centering
	\includegraphics[max width=\textwidth,keepaspectratio=true]{imgs-70-prakticka/cetnost-slov}
	\caption[Srovnání relativních četností výskytu 10 nejčastějších slov v analyzovaných překladech.]{Srovnání relativních četností výskytu 10 nejčastějších slov v analyzovaných překladech. \textit{Zdroj:~vlastní.}}
	\label{fig:word-freq}
\end{figure}

V grafu vidíme, že relativní četnosti v analyzovaných překladech dosahují vyšších hodnot než ve frekvenčních seznamech. To je způsobeno jednak rozdílnou velikostí zkoumaného souboru textů a jednak vzájemnou podobností (a tím nižší variabilitou) českých překladů. 

Vůbec nejčastějším slovem v českých překladech byla předložka \enquote{v}. Ta se v referenčních seznamech vždy umístila do čtvrtého místa. Ve frekvenčním seznamu \textit{Jelínek 1961} skončila na čtvrtém místě, v seznamu \textit{Čermák 2004} se umístila na třetím místě a ve frekvenčním seznamu \textit{Těšitelová 1983} skončila dokonce druhá. \enquote{V} se nejčastěji vyskytuje v překladu Jaroslava Vrchlického z roku 1881 ($36\times$), se $34$ výskyty následuje Vrchlického překlad z roku 1890. Na třetím místě figuruje překlad Dagmar Wagnerové ($33$ výskytů). Následují Reslerovy překlady z let 1948 ($32$ výskytů) a 1956 ($30$ výskytů). Oba překlady Vrchlického a Šembery se umístily do pátého místa. Přestože se od sebe jednotlivé verze odlišují, slovo \enquote{v} je v každé z nich zastoupeno nadprůměrně.

Nadprůměrně je zastoupeno také slovo \enquote{jak}, kterému v Jelínkově seznamu patří 29. a v~Čermákově dokonce až 36. místo. Nadprůměrně jsou zastoupeny také slova \enquote{havran} a \enquote{pták}, což pochopitelně souvisí s charakterem testovaného souboru. V tabulce \ref{tab:word-raven-bird-compare} je uvedeno pořadí výskytu slov \enquote{havran} a \enquote{pták} v seznamech, které vznikly na základě analýzy větších a variabilnějších souborů textu\footnote{Ze seznamu Těšitelová 1983 má autor k dispozici pouze deset nejfrekventovanějších slov uvedených v \cite[str.~19]{Tesitelova1987}, proto není v tabulce uveden.}. 

\begin {table}[H]
	\caption {Srovnání pořadí výskytu slov \enquote{havran} a \enquote{pták}}
	\label{tab:srovnani-poradi-vyskytu-slov} 

	\begin{center}
		\begin{tabular}{{l||c|c}}
		\hline
		\bfseries  & \bfseries Pořadí v seznamu Jelínek 1961 & \bfseries Pořadí v seznamu Čermák 2004 \\
		    \hline \hline
		   \bfseries Havran    & 7895. &   5446.   \\\hline
		   \bfseries Pták& 991. &     2295.    \\\hline
		\end{tabular}
	\end{center}
	\label{tab:word-raven-bird-compare}
\end{table}

\subsection{Porovnání českých překladů s originálem}

Pro porovnání českých překladů s originálem budou parametry analýzy nastaveny stejně jako při porovnání s frekvenčními seznamy v kapitole \ref{chap:word-freq-list}. Navíc budou zavedena dvě nová pravidla zodpovědná za nahrazení anglického \enquote{raven} českým \enquote{havran} a anglického \enquote{bird} českým \enquote{pták}. Tím bude usnadněno porovnání jednotlivých překladů s originálním textem.

Na obrázku \ref{fig:word-freq-compare} je znázorněno srovnání četností výskytů slov \enquote{havran} (respektive \enquote{raven}) a pták (\enquote{bird}) v českých překladech a v originále. 

\begin{figure}[h!]
	\centering
	\includegraphics[max width=\textwidth,keepaspectratio=true]{imgs-70-prakticka/cetnost-slov-orig}
	\caption[Srovnání četností výskytů slov \enquote{havran} a \enquote{pták} v originále a v 18 českých překladech.]{Srovnání četností výskytů slov \enquote{havran} (červeně) a \enquote{pták} (modře) v originále a v 18 českých překladech v pořadí dle přílohy \ref{appendix:preklady}. \textit{Zdroj:~vlastní.}}
	\label{fig:word-freq-compare}
\end{figure}

V anglickém originálu se obě slova vyskytla desetkrát. Z českých překladů dosahuje stejných četností pouze překlad Jiřího Taufera. Ve Vrchlického překladu z roku 1881 najdeme \enquote{havrana} stejně jako v originále $10\times$, \enquote{pták} je však obsažen pouze devětkrát. Za povšimnutí stojí, že ve verzi vydané roku 1890 se \enquote{pták} vyskytuje již $10\times$, ovšem počet výskytů \enquote{havrana} klesl na osm. Od originálu o dva výskyty se lišící jsou i překlady Rudolfa Havla a Augustina Eugena Mužíka -- v obou se shodně nachází \enquote{havran} jedenáctkrát a \enquote{pták} devětkrát. Naopak nejméně podobný originálu je překlad Vratislava Kazimíra Šembery -- v jeho překladu se \enquote{havran} vyskytl $16\times$, naopak \enquote{pták} pouze třikrát.

\section{Agregace}

Analýzu agregace je možné provést jak pro jeden, tak pro všechny otevřené texty. Analyzovány jsou veršové páry pro vzdálenosti od jednoho do deseti veršů. Analýza začíná vytvořením množin $A$ a $B$ pro všechny verše analyzované básně. Poté je báseň procházena od prvního do posledního verše. Aktuální verš je považován za pivot. Od pivotu jsou vypočítány velikosti průniků množin $A_{pivot}$ a $B_{pivot}$ s množinami následujících veršů $A_{pivot + x}$  s $B_{pivot + x}$. 

Tyto údaje jsou v průběhu analýzy pouze uloženy, jejich vyhodnocení probíhá až později při zobrazení výsledků. Na obrázku \ref{fig:agregace-vrchlicky-1890} je zobrazen report z agregace Vrchlického překladu z roku 1890. V tabulce jsou zobrazeny hodnoty podobnosti $\overline{S}_i$ vypočítané jako průměr ze všech hodnot pro daný posun $L_i$ a aproximované hodnoty $\hat{S}_i$, které byly vypočítány metodou nejmenších čtverců.

Pod tabulkou je uvedena aproximační funkce $S$, která je použita pro vyjádření tzv. \textit{křivky fonické podobnosti veršů}. Ta vyjadřuje závislost fonické podobnosti veršů $S$ na jejich vzájemné vzdálenosti $L$ \cite[str.~78]{Wimmer2003}. Pod aproximační funkcí $S$ je uvedena hodnota koeficientu determinace $D$, která udává podíl rozptylu závislé proměnné, který se podařilo aproximací vysvětlit. Na základě hodnoty koeficientu determinace $D$ je možné předběžně přijmout hypotézu, že aproximační funkce $S$ vhodně vyjadřuje závislost fonické podobnosti veršů v závislosti na jejich vzdálenost \cite[str.~79]{Wimmer2003}. 

Do grafu pod tabulkou jsou na ose $x$ vyneseny hodnoty $L_i$, osa $y$ zachycuje hodnoty  $\overline{S}_i$ a $\hat{S}_i$. V grafu jsou zakresleny jak průměrné hodnoty $\overline{S}_i$, tak křivka aproximující naměřené hodnoty. 

\begin{figure}[H]
	\centering
	\includegraphics[max width=\textwidth,keepaspectratio=true]{imgs-70-prakticka/aggregation_vrchlicky_1890}
	\caption[Agregace pro Vrchlického překlad z roku 1890]{Agregace pro Vrchlického překlad z roku 1890. \textit{Zdroj:~vlastní.}}
	\label{fig:agregace-vrchlicky-1890}
\end{figure}

Pro všechny z osmnácti analyzovaných překladů je možné pozorovat klesající podobnost veršů vzhledem k jejich rostoucí vzdálenosti. Každý z překladů však vykazoval odchylku od tohoto trendu, a to v závislosti na struktuře básně. V překladech ctících délku strofy originálu (tj. šest veršů v jedné strofě) byly naměřené hodnoty $\overline{S}_6$ výrazně vyšší než aproximované hodnoty $\hat{S}_6$. Pro překlady D.~Wagnerové, A.~E.~Mužíka a J.~Vrchlického z roku 1890, jejichž strofy jsou tvořeny jedenácti verši, byla odlišná vzdálenost $L_2$, což ukazuje i obrázek \ref{fig:agregace-vrchlicky-1890}.

Pro ani jeden z analyzovaných překladů se nepodařilo naměřené hodnoty dostatečně přesně aproximovat. Koeficienty determinace byly vyšší u překladů se strofami tvořenými jedenácti verši. Nejvyššího koeficientu determinace $D$ bylo dosaženo u překladu Jaroslava Vrchlického z roku 1890 ($D = 0{,}504$), následoval překlad D.~Wagnerové s $D = 0{,}391$ a Mužíkův překlad, u nějž byla hodnota $D$ rovna $0{,}307$. Z ostatních překladů byla pouze u překladu K.~Reslera z roku 1956 hodnota $D$ vyšší něž $0{,}1$ (konkrétně $0{,}106$). Hodnoty $D$ se u ostatních překladů pohybovaly pod hranicí $0{,}1$.

\section{Asonance}

Analýza asonance patří mezi analýzy, které mohou být prováděny pouze nad všemi otevřenými texty. Program umožňuje testování dvou hypotéz $H_0$:
\begin{enumerate}
\item $H_0$: Střední hodnoty jsou vzhledem k různým posunům stejné.
\item $H_0$: Střední hodnoty jsou vzhledem ke konstantnímu posunu stejné.
\end{enumerate}
Obě hypotézy jsou podrobně popsány v kapitole \ref{chap:app_asonance}.

Pro testování českých překladů Havrana dává smysl provést test pouze první z hypotéz, neboť testování druhé hypotézy vyžaduje rozdělení analyzovaných textů na jednotlivé výběry (i když je samozřejmě možné rozdělit české překlady do jednotlivých výběrů např. podle data publikace). Druhá hypotéza $H_0$ najde uplatnění spíše při srovnání různojazyčných překladů, větších básnických souborů a podobně.

První z hypotéz testuje, zda se v analyzovaných textech liší asonance pro různé posuny. Testovány jsou posuny o jeden až patnáct vokálů, přičemž maximální posun může být i snížen, pokud je některý z testovaných textů tak krátký, že v něm není možné realizovat požadovaný počet posunů.

Naměřené hodnoty asonance jsou následně zobrazeny v reportu. Ukázku reportu pro české překlady je možné si prohlédnout v příloze \ref{appendix:asonance}. Hodnoty pro jednotlivé texty jsou zobrazeny v tabulce \enquote{Výsledky asonance}. Ta je ve výchozím stavu skryta. Následuje sekce zobrazující informace o Bartlettově testu. Bartlettův test je použit pro testování shody rozptylů. Hypotézy Bartlettova testu zní $H_0: \sigma^2_1 = \sigma^2_2 = \dots = \sigma^2_n$ proti $H_A: non H_0$.

Další sekce obsahuje jednu z metod pro analýzu rozptylu. Není-li nulová hypotéza Bartlettovým testem zamítnuta, jedná se o \textit{ANOVU}. Při zamítnutí $H_0$ v Bartlettově testu je použit \textit{Kruskallův-Wallisův test}.

Nulová hypotéza pro Anovu je dána jako $H_0: \mu_1 = \mu_2 = \dots = \mu_n$, alternativní hypotéza $H_A$ jako $H_A:  \mbox{non } H_0$. Hypotéza testovaná Kruskallovým-Wallisovým testem zní $H_0: F_1(x) = F_2(x) = \dots = F_n(x)$ pro všechna $x$. Naproti ní stojí alternativní hypotéza $H_A: \mbox{non } H_0$. Jak Anova tak Kruskallův-Wallisův test testují, zda se mezi jednotlivými posuny nacházejí takové, které se významně liší.

V případě zamítnutí $H_0$ v jednom z testů analýzy rozptylu je v reportu zobrazena sekce obsahující lišící se posuny. Ty jsou zjišťovány Schéffeho metodou. Sekce obsahuje jak tabulku s~konkrétními hodnotami testovacího kritéria, tak slovně vypsané lišící se posuny.

V českých překladech Havrana se významně lišilo šestnáct dvojic uvedených v tabulce \ref{tab:asonance-lisici-se}.
\begin {table}[H]
	\caption {Přehled lišících se asonančních posunů v českých překladech} 
	\label{tab:asonance-lisici-se} 

	\begin{center}
		\begin{tabular}{|l|l|l|l|}
		\hline Posun o 1 a o 7   & Posun o 3 a o 7 & Posun o 5 a o 14 & Posun o 8 a o 11  \\ 
		\hline Posun o 1 a o 8   & Posun o 3 a o 8 & Posun o 7 a o 11 & Posun o 8 a o 12  \\ 
		\hline Posun o 1 a o 14  & Posun o 5 a o 7 & Posun o 8 a o 9  & Posun o 8 a o 13  \\ 
		\hline Posun o 2 a o 8   & Posun o 5 a o 8 & Posun o 8 a o 10 & Posun o 11 a o 14 \\ 
		\hline 
		\end{tabular} 
	\end{center}
\end{table}

\section{Aliterace}

Asonance spolu s denotační analýzou patří mezi ty analýzy, které mohou být provedeny pouze na jednom textu. Při analýze samotné se pouze shromáždí podklady pro další zpracování, žádné výpočty nejsou během analyzování prováděny.

Při analýze jsou postupně procházeny všechny verše analyzované básně. Každý verš je rozdělen na jednotlivá slova a z každého slova je jeho první písmeno zařazeno mezi počáteční písmena daného verše, přičemž je uchována i informace, kolikrát se to které písmeno v rámci analyzovaného verše vyskytlo. Na konci zpracování daného verše jsou odstraněna ta počáteční písmena, která se vyskytla méně než dvakrát.

Report z analýzy aliterace obsahuje tabulku, jejíž každý řádek reprezentuje jeden verš analyzované básně. Pro každý verš jsou uvedeny fonémy, které v daném verši aliterovaly, a počet slov daného verše. Ve čtvrtém sloupci tabulky je uveden počet slov, které začínají aliteračními fonémy. V předposledním sloupečku je uvedena pravděpodobnost výskytu nalezené aliterace. Poslední sloupeček uvádí tzv. \textit{aliterační charakter} definovaný v \cite[str.~60]{Wimmer2003} jako

\[KA = \left\{
  \begin{array}{lr}
    100 \times [ \alpha - P( \xi \geq x)],& \text{ pro } \alpha >  P( \xi \geq x)\\
    0,& \text{ pro } \alpha \leq  P( \xi \geq x)
  \end{array}
\right.
\]

Nad tabulkou je zobrazen \textit{aliterační charakter básně} daný jako průměr všech aliteračních charakterů jednotlivých veršů. Výpočet hodnot $KA$ je možné ovlivnit volbou hladiny významnosti $\alpha$ z intervalu $\langle 0{,}01; 1 \rangle$.

Na obrázku \ref{fig:alliteration} je zobrazen začátek reportu z analýzy aliterace pro originální text E. A. Poea The Raven.\newpage

\begin{figure}[H]
	\centering
	\includegraphics[max width=\textwidth,keepaspectratio=true]{imgs-70-prakticka/alliteration}
	\caption[Výsledek analýzy aliterace pro prvních 11 veršů originálního textu The Raven E.A. Poea]{Výsledek analýzy aliterace pro prvních 11 veršů originálního textu The Raven E.A. Poea. \textit{Zdroj:~vlastní.}}
	\label{fig:alliteration}
\end{figure}

Tabulka \ref{tab:alliteration-compare} obsahuje srovnání aliterace pro analyzované české překlady při $\alpha = 0{,}05$. Překlady jsou v tabulce seřazeny vzestupně dle hodnoty $\KA$. Strofy překladů Mužík, Vrchlický 1890 a Wagnerová jsou tvořeny jedenácti verši namísto šesti, jako je tomu u originálního textu a ostatních překladů. Pro srovnání: $\KA_{\text{original}} = 3{,}79759$.

\begin {table}[H]
	\caption {Aliterační charakter $KA$ českých překladů při $\alpha = 0{,}05$}
	\label{tab:alliteration-compare} 

	\begin{center}
		\begin{tabular}{{lr||lr||lr}}
			\hline

			\bfseries Text & \bfseries $KA_{\text{báseň}}$ & \bfseries Text & \bfseries $KA_{\text{báseň}}$ & \bfseries Text & \bfseries $KA_{\text{báseň}}$ \\
				\hline \hline
				Mužík          & $1{,}58133$ & Resler 1956 & $3{,}33302$  & Šembera        & $3{,}57143$ \\ \hline
				Vrchlický 1890 & $1{,}76623$ & Taufer      & $3{,}36161$  & Dostál-Lutinov & $3{,}58246$ \\ \hline
				Wagnerová      & $1{,}98075$ & Resler 1948 & $3{,}48301$  &	Vrchlický 1881 & $3{,}64690$ \\ \hline
				Čapek          & $3{,}09970$ & Babler      & $3{,}49542$  & Kadlec         & $3{,}66010$ \\ \hline
				Nezval         & $3{,}14763$ & Slavík      & $3{,}53020$   & Stoklas       & $3{,}87856$ \\ \hline
				Havel          & $3{,}28966$ & Černý       & $3{,}56506$  & Bejblík        & $3{,}99667$ \\ \hline
		\end{tabular}
	\end{center}
\end{table}

\section{Denotační analýza}
\label{chap:denotation-analysis} 

V rámci denotační analýzy byly etablovány hřeby pro čtrnáctou sloku všech osmnácti českých překladů. K omezení délky textů na pouze 14. sloku bylo přistoupeno zejména z důvodu složitosti takové analýzy. Jak uvádí Wimmer v \cite[str.~300]{Wimmer2003}: 

\begin{quote}
Treba poznamenať, že táto analýza je relatívne jednoduchá v krátkých textoch. Pri dlhých textoch bude pravdepodobne potrebné text najprv zakódovať, aby sa dal mechanicky spracovať, lebo pravdebodobnosť omylu stúpá od vety k vete.
\end{quote}

Text čtrnácté strofy v anglickém originálu E. A. Poea zní:

\begin{verse}
 Then, methought, the air grew denser, perfumed from an unseen censer \\
 Swung by seraphim whose foot-falls tinkled on the tufted floor.\\
 “Wretch,” I cried, “thy God hath lent thee — by these angels he hath sent thee \\
 Respite — respite and nepenthe, from thy memories of Lenore; \\
 Quaff, oh quaff this kind nepenthe and forget this lost Lenore!”\\
 Quoth the Raven “Nevermore.” 
\end{verse}
\begin{flushright}
Zdroj: \cite[str.~65]{Poe1990}
\end{flushright}

Na CD přiloženém k této diplomové práci jsou umístěny soubory obsahující jednak analyzované čtrnácté sloky jednotlivých překladů a jednak samotné rozdělení těchto slok do hřebů. Jak bylo popsáno v kapitole \nameref{chap:app-denotation-analysis}, je možné tyto soubory použít k vlastní analýze. Elektronická příloha také obsahuje koincidenční a dererministicko-pravděpodobnostní grafy pro všechny překlady. Ty nejzajímavější jsou uvedeny přímo v této práci.

\subsection{Parametr $\alpha$}

Zásadní vliv na uspořádání výsledného grafu má zvolená hodnota hladiny významnosti $\alpha$. Výchozí hodnota hladiny významnosti je nastavena na $\alpha=0{,}1$, přičemž program umožňuje její hodnotu měnit v intervalu $\langle 0{,}001;1\rangle$ s krokem $0{,}001$.

Pokud je zvolená hodnota $\alpha$ příliš nízká, obsahuje výsledný graf mnoho izolovaných vrcholů. Čím je naopak hodnota $\alpha$ vyšší, tím je vyšší i počet hran grafu. Pro zvolení vhodné hodnoty je třeba experimentovat. 

Následující obrázky \ref{fig:denotation-resler-005}, \ref{fig:denotation-resler-007}, \ref{fig:denotation-resler-034}  zobrazují koincidenční grafy vytvořené na základě denotační analýzy čtrnácté strofy Reslerova překladu. Pro všechna $\alpha \leq 0{,}06$ tvoří graf pouze izolované vrcholy (hodnota indexu nespojitosti je tedy vyšší než $0{,}06$). Při hodnotě $\alpha = 0{,}07$ dojde k vytvoření hran mezi vrcholy (hřeby) $17$, $21$ a $22$. K další redukci počtu komponent dojde při zvýšení $\alpha$ z na $0{,}17$, kdy se počet komponent sníží z $28$ na $10$ (naopak počet hran vzroste ze tří na $42$). Při $\alpha = 0{,}2$ se počet komponent sníží na devět. V tu chvíli zbývají v grafu pouze dva izolované vrcholy -- $2$ a $7$. Při $\alpha = 0{,}34$ se počet komponent sníží na tři. Vrcholy $2$ a $7$ se stanou součástí komponenty K1 tvořenou vrcholy $1$ až $15$ (s výjimkou hřebu č. $10$). Při $\alpha = 0{,}5$ dojde k poslední redukci počtu komponent, kdy se propojí komponenty K1 a K2. Ke spojení těchto dvou komponent i přes zvyšování hodnoty $\alpha$ nedojde, neboť index dosažitelnosti je roven $\infty$.

\begin{figure}[H]
	\centering
	\includegraphics[max width=\textwidth,height=150px,keepaspectratio=true]{imgs-70-prakticka/denotation-resler-017}
	\caption[Koincidenční graf Reslerova překladu pro $\alpha = 0{,}17$.]{Koincidenční graf Reslerova překladu pro $\alpha = 0{,}17$. \textit{Zdroj:~vlastní.}}
	\label{fig:denotation-resler-005}
\end{figure}

\begin{figure}[H]
	\centering
	\includegraphics[max width=\textwidth,keepaspectratio=true]{imgs-70-prakticka/denotation-resler-034}
	\caption[Koincidenční graf Reslerova překladu pro $\alpha = 0{,}34$.]{Koincidenční graf Reslerova překladu pro $\alpha = 0{,}34$. \textit{Zdroj:~vlastní.}}
	\label{fig:denotation-resler-034}
\end{figure}

\begin{figure}[H]
	\centering
	\includegraphics[max width=\textwidth,keepaspectratio=true]{imgs-70-prakticka/denotation-resler-05}
	\caption[Koincidenční graf Reslerova překladu pro $\alpha = 0{,}07$.]{Koincidenční graf Reslerova překladu pro $\alpha = 0{,}07$. \textit{Zdroj:~vlastní.}}
	\label{fig:denotation-resler-007}
\end{figure}

V tabulce \ref{tab:prehled-indexu-grafy} jsou uvedeny indexy nespojitosti, neizolovanosti a dosažitelnosti všech osmnácti analyzovaných překladů:

\begin {table}[H]
	\caption {Přehled indexů nespojitosti, neizolovanosti a dosažitelnosti} 
	\label{tab:prehled-indexu-grafy} 

	\begin{center}
		\begin{tabular}{{l|r|r|r|c}}

		\hline

		\multirow{2}{*}{\bfseries Překlad} &
		\multicolumn{3}{c|}{\bfseries Index} &
		\multirow{2}{*}{\bfseries \parbox{3cm}{\centering Minimální počet komponent}} \\
		\cline{2-4}
			 & \bfseries nespojitosti & \bfseries neizolovanosti & \bfseries dosažitelnosti \\
		    \hline \hline
		   E. A. Poe         & $0{,}067$        & $0{,}400$         &    $\infty$   & 2 \\ \hline \hline
		   V. K. Šembera     & $0{,}067$        & $0{,}333$         &    $0{,}500$  & 1 \\ \hline
		   J. Vrchlický 1881 & $0{,}167$        & $0{,}400$         &    $0{,}667$  & 1 \\ \hline
		   J. Vrchlický 1890 & $0{,}091$        & $0{,}182$         &    $\infty$   & 3 \\ \hline
		   A. E. Mužík       & $0{,}018$        & $0{,}273$         &    $\infty$   & 3 \\ \hline
		   K. Dostál-Lutinov & $0{,}067$        & $0{,}400$         &    $\infty$   & 2 \\ \hline
		   V. Nezval         & $0{,}067$        & $0{,}500$         &    $\infty$   & 2 \\ \hline
		   O. F. Babler      & $0{,}067$        & $0{,}500$         &    $\infty$   & 2 \\ \hline
		   J. Taufer         & $0{,}167$        & $0{,}667$         &    $\infty$   & 2 \\ \hline
		   E. Stoklas        & $0{,}067$        & $0{,}333$         &    $\infty$   & 3 \\ \hline
		   D. Wagnerová      & $0{,}018$        & $0{,}273$         &    $\infty$   & 4 \\ \hline
		   R. Havel          & $0{,}067$        & $0{,}400$         &    $\infty$   & 2 \\ \hline
		   J. B. Čapek       & $0{,}067$        & $0{,}500$         &    $\infty$   & 2 \\ \hline
		   K. Resler 1948    & $0{,}067$        & $0{,}333$         &    $\infty$   & 2 \\ \hline
		   K. Resler 1956    & $0{,}067$        & $0{,}200$         &    $\infty$   & 3 \\ \hline
		   R. Černý          & $0{,}067$        & $0{,}400$         &    $\infty$   & 2 \\ \hline
		   I. Slavík         & $0{,}067$        & $0{,}400$         &    $\infty$   & 2 \\ \hline
		   S. Kadlec         & $0{,}167$        & $0{,}400$         &    $\infty$   & 2 \\ \hline
		   A. Bejblík        & $0{,}067$        & $0{,}333$         &    $0{,}500$  & 1 \\ \hline
		   \bfseries Průměr  & $\approx 0{,}079$ & $\approx 0{,}380$&               &   \\ \hline
		\end{tabular}
	\end{center}
\end{table}

Průměrná hodnota indexu nespojitosti českých překladů je $\approx 0{,}08$, index neizolovanosti dosahuje průměrné hodnoty $\approx 0{,}38$. 
Většina z indexů dosažitelnosti uvedených v tabulce \ref{tab:prehled-indexu-grafy} dosahuje hodnoty nekonečno. To znamená, že ať je hodnota $\alpha$ jakákoliv, nebude graf nikdy tvořen pouze jednou komponentou. Samostatná komponenta je typicky tvořena posledním veršem, který na rozdíl od zbytku strofy odkazuje na havrana.

G. Wimmer a kolektiv uvádí v \cite[str.~309]{Wimmer2003}: 

\begin{quote}
Tieto tri uvedené indexy sú pre daný text charakteristické a možno ich použiť priamo na porovnanie textov s približne rovnakým rozsahom\ldots
\end{quote}

Podmínka na stejný rozsah analyzovaných textů je splněna, neboť je analyzována vždy stejná strofa. Pokud bychom zkoumali, které překlady jsou si na základě indexů nespojitosti, neizolovanosti a dosažitelnosti nejbližší, dospěli bychom k těmto závěrům:

Obecně lze říci, že jsou si jednotlivé překlady podobné. Ze všech osmnácti analyzovaných strof je pět překladů charakterizováno unikátními indexy: překlad Kamila Reslera vydaný roku 1948, překlady Jaroslava Vrchlického z let 1881 a 1890, překlad Svatoupluka Kadlece a překlad Jiřího Taufera. Hodnoty měřených indexů ostatních překladů již unikátní nejsou. V tabulce \ref{tab:prehled-stejne-indexy} jsou uvedeny skupiny překladů se stejnými indexy. O překladech v jedné skupině můžeme tvrdit, že jsou si vzájemně podobnější než překlady z jiných skupin.

\begin {table}[H]
	\caption {Přehled indexů nespojitosti, neizolovanosti a dosažitelnosti} 
	\label{tab:prehled-stejne-indexy} 

	\begin{center}
		\begin{tabular}{{l|l}}
		\hline

		\bfseries \parbox[c][1.2cm]{5.5cm}{\centering Index nespojitosti/ neizolovanosti/dosažitelnosti} &
		\bfseries Překlady \\
			\hline \hline
			$0{,}018$/$0{,}273$/--                 & A. E. Mužík, D. Wagnerová                          \\ \hline
			$0{,}067$/$0{,}333$/$0{,}5$            & V. K. Šembera, A. Bejblík                          \\ \hline
			$0{,}067$/$0{,}333$/--                 & E. Stoklas, K. Resler 1948                         \\ \hline
			$0{,}067$/$0{,}400$/--                 & \textbf{E. A. Poe}, K. Dostál-Lutinov, R. Havel, R. Černý, I. Slavík   \\ \hline
			$0{,}067$/$0{,}500$/--                 & V. Nezval, O. F. Babler, J. B. Čapek               \\ \hline
		\end{tabular}
	\end{center}
\end{table}

\subsection{Denotační charakteristiky textu}

Mezi sledované denotační charakteristiky patří počet veršů $N_v$, počet denotačních elementů $L$, počet etablovaných hřebů $N_h$, kardinální číslo jádra $|\kern 2pt.\kern 2pt|$, kompaktnost textu $K$, centralizovanost $R$ a MacInthosův index $R_{rel}$. Jednotlivé charakteristiky analyzovaných strof jsou uvedeny v~tabulce \ref{tab:prehled-charakteristiky}.



\begin {table}[H]
	\caption {Přehled denotačních charakteristik} 
	\label{tab:prehled-charakteristiky} 

	\begin{center}
		\begin{tabular}{{l|c|c|c|c|c|c|c}}
		\hline

		\bfseries Překlad & \bfseries $N_v$ & \bfseries $L$ & \bfseries $N_h$ & \bfseries \parbox[c][1.2cm]{4cm}{\centering $|\kern 2pt.\kern 2pt|$\\(jádrové hřeby)} & \bfseries $K$ & \bfseries $R$ & \bfseries $R_{rel}$ \\
			\hline \hline
		   E. A. Poe         & $6$    & $53$ & $21$ & \parbox[c][1.1cm]{3cm}{\centering28\\(2 3 4 7 8 11 20)}         & $0{,}62$ & $0{,}071$ & $0{,}75$ \\ \hline\hline
		   V. K. Šembera     & $6$    & $55$ & $24$ & \parbox[c][1.1cm]{3cm}{\centering28\\(1 4 8 11 14 18)}          & $0{,}57$ & $0{,}073$ & $0{,}74$ \\ \hline
		   J. Vrchlický 1881 & $6$    & $50$ & $29$ & \parbox[c][1.1cm]{4cm}{\centering19\\(3 10 18 21 22)}           & $0{,}43$ & $0{,}056$ & $0{,}78$ \\ \hline
		   J. Vrchlický 1890 & $11$   & $51$ & $34$ & \parbox[c][1.1cm]{4cm}{\centering16\\(3 10 19 22 23 30)}        & $0{,}34$ & $0{,}039$ & $0{,}82$ \\ \hline
		   A. E. Mužík       & $11$   & $53$ & $32$ & \parbox[c][1.1cm]{4cm}{\centering21\\(2 11 13 16 30)}           & $0{,}40$ & $0{,}061$ & $0{,}77$ \\ \hline
		   K. Dostál-Lutinov & $6$    & $49$ & $29$ & \parbox[c][1.1cm]{4cm}{\centering20\\(2 4 7 8 10 14)}           & $0{,}42$ & $0{,}048$ & $0{,}80$ \\ \hline
		   V. Nezval         & $6$    & $56$ & $38$ & \parbox[c][1.1cm]{4cm}{\centering16\\(6 11 17 18 28 34)}        & $0{,}33$ & $0{,}035$ & $0{,}83$ \\ \hline
		   O. F. Babler      & $6$    & $58$ & $30$ & \parbox[c][1.1cm]{4cm}{\centering31\\(1 3 10 15 19 21 27)}      & $0{,}49$ & $0{,}061$ & $0{,}77$ \\ \hline
		   J. Taufer         & $6$    & $54$ & $36$ & \parbox[c][1.1cm]{4cm}{\centering18\\(1 5 12 19 33)}            & $0{,}34$ & $0{,}051$ & $0{,}79$ \\ \hline
		   E. Stoklas        & $6$    & $54$ & $31$ & \parbox[c][1.1cm]{4cm}{\centering23\\(5 8 10 12 14 17 29)}      & $0{,}43$ & $0{,}050$ & $0{,}79$ \\ \hline
		   D. Wagnerová      & $11$   & $55$ & $32$ & \parbox[c][1.1cm]{4cm}{\centering22\\(3 4 7 9 10 16 29)}        & $0{,}43$ & $0{,}053$ & $0{,}78$ \\ \hline
		   R. Havel          & $6$    & $55$ & $31$ & \parbox[c][1.1cm]{4cm}{\centering26\\(1 5 6 10 16 19 28)}       & $0{,}44$ & $0{,}057$ & $0{,}77$ \\ \hline
		   J. B. Čapek       & $6$    & $47$ & $29$ & \parbox[c][1.1cm]{4cm}{\centering19\\(1 3 5 11 14 20)}          & $0{,}39$ & $0{,}048$ & $0{,}80$ \\ \hline
		   K. Resler 1948    & $6$    & $54$ & $30$ & \parbox[c][1.1cm]{4cm}{\centering22\\(1 2 10 12 27)}            & $0{,}45$ & $0{,}064$ & $0{,}76$ \\ \hline
		   K. Resler 1956    & $6$    & $59$ & $31$ & \parbox[c][1.1cm]{4cm}{\centering26\\(1 2 6 10 13 28)}          & $0{,}48$ & $0{,}066$ & $0{,}76$ \\ \hline
		   R. Černý          & $6$    & $58$ & $35$ & \parbox[c][1.1cm]{4cm}{\centering22\\(2 3 5 11 18 24 31)}       & $0{,}40$ & $0{,}044$ & $0{,}80$ \\ \hline
		   I. Slavík         & $6$    & $54$ & $35$ & \parbox[c][1.1cm]{4cm}{\centering17\\(3 8 10 19 24 28 32)}      & $0{,}36$ & $0{,}036$ & $0{,}82$ \\ \hline
		   S. Kadlec         & $6$    & $60$ & $33$ & \parbox[c][1.1cm]{4cm}{\centering26\\(1 4 6 10 17 25 30)}       & $0{,}46$ & $0{,}057$ & $0{,}77$ \\ \hline
		   A. Bejblík        & $6$    & $55$ & $30$ & \parbox[c][1.1cm]{4cm}{\centering27\\(2 5 8 11 15 16 17 20 28)} & $0{,}46$ & $0{,}050$ & $0{,}79$ \\ \hline
		\end{tabular}
	\end{center}
\end{table}
\end{document}
