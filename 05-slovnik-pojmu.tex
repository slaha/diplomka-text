\newglossaryentry{denotat}
{
	name=denotát,
	description={Něco označovaného či něco, k čemu je odkazováno, konkrétní význam slova}
}

\newglossaryentry{denotace}
{
	name=denotace,
	description={Vztah výrazu k \glslink{denotat}{denotátu}}
}

\newglossaryentry{substantivum}
{
	name=substantivum,
	description={Podstatné jméno}
}

\newglossaryentry{slovesnyvid}
{
	name=slovesný vid,
	description={Slovesný vid určuje, zda sloveso vyjadřuje ukončený nebo ukončitelný děj (tzv. \textbf{dokonavý vid}), nebo děj, který ukončený být nemůže nebo je časově neohraničený (tzv. \textbf{nedokonavý vid})}
}

\newglossaryentry{lemma}
{
	name=lemma,
	description={Lexikální jednotka se všemi tvary reprezentovaná základním tvarem, heslové slovo ve slovníku}
}

\newglossaryentry{tema_rema}
{
	name=téma a réma,
	description={Téma a réma jsou složky teorie aktuálního členění větného. Téma je to, o čem se mluví, a réma to, co se o tématu říká }
}

\newglossaryentry{bigram}
{
	name=bigram,
	description={Bigram nebo digram je každá posloupnost dvou sousedních znaků v textovém řetězci}
}

\newacronym{gui}{GUI}{\textbf{G}raphical \textbf{U}ser \textbf{I}nterface -- grafické uživatelské rozhraní}


\newacronym{csv}{CSV}{\textbf{C}omma \textbf{S}eparated \textbf{V}alues -- hodnoty oddělené čárkou}


\newacronym{api}{API}{\textbf{A}pplication \textbf{P}rogramming \textbf{I}nterface -- rozhraní pro programování aplikací. Jedná se o soubor funkcí, metod a datových typů, které může programátor využít při vytváření aplikace}
